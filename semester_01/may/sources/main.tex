\documentclass{article}
\usepackage[utf8]{inputenc}

\usepackage[T2A]{fontenc}
\usepackage[utf8]{inputenc}
\usepackage[russian]{babel}

\usepackage{multienum}
\usepackage{geometry}

\geometry{
    left=1cm,right=1cm,
    top=2cm,bottom=2cm
}


\title{История}
\author{Лисид Лаконский}
\date{May 2023}

\newtheorem{definition}{Определение}

\begin{document}
\raggedright

\maketitle
\tableofcontents
\pagebreak

\section{История — 18.05.2023}

\subsection{Россия в советский период, 1917–1945 годы}

В 1917 году происходит великая российская революция. \textbf{25 октября 1917 года — вооруженное восстание большевиков} с целью захвата власти и свержения Временного правительства.

Влияние на мир не меньшее, чем великая французская революция. Большее влияние, чем американская революция (война за независимость).

Основные причины:

\begin{enumerate}
    \item \textbf{Высокий уровень бедности} — до 40\% всего населения (свыше 180 млн человек) были бедными.
    \item Государство в предреволюционные годы в России развивалось намного медленнее, чем общество. \textbf{Государство было более архаическим, чем общество}. \\
    Очень развитая, утонченная интеллигенция — крестьянство, в основном неграмотное и необразованное, живет архаическими представлениями, общинный уклад, традиционализм, глава крестьянской семьи решает все. Выучивание — воспроизведение, передача традиций. \\
    \textbf{Слишком неоднородный, противоречивый, характер социального, экономического, культурного развития государства}.
    \item \textbf{Имперский характер страны}. Тренд на эмансипацию людей. \\
    Российская империя Романовых поликультурна, полирелигиозна: католики, протестанты, буддисты, более примитивные религии. Националистическая политика последних правительств провоцировала противоречия между метрополией и окраиной. Россия вновь возвращается в средние века. \\
    Империи стабильны, пока существует сильная метрополия, гарантирующая порядок. Распад российской империи был частью мирового исторического процесса. Влияние первой мировой войны. Глобальный исторический тренд. \\
    Ввиду непродуманной политики противоречия между центром и окраиной стали слишком сильны.
    \item \textbf{Участие России в первой мировой войне}. Россия оказалась самым слабым звеном в мировой цепи империализма. Большинство населения не хотело этой войны, не хотело умирать, не понимало ее цель. \\
    Начало войны — патриотический подъем. С одной стороны, царь был уверен, что нельзя дальше отступать — в итоге для него все это кончилось революцией. \\
    1914 год — относительно спокойно с точки отношения общества, 1915 год — второй этап войны на восточном фронте, большое отступление русской армии — снарядный голод. Дух русского солдата, отношения общества к событиям надломлены. \\
    Некомпетентность генералов, губернаторов, императора — \textbf{десакрализация} авторитета царя. \\
    По политическим взглядам большинство людей было монархистами. Николай II оказался очень плохим политиком. С начала войны хотел занять должность верховного главнокомандующего, в 1915 году сам себя им назначает. Одновременно выполнял функцию, как мог, верховного главнокомандующего, стал редко бывать в столице — сыграло негативную роль в последующих событиях. \\
    Окружение царя было даже политически дальновидней, чем царь. \\
    1916 год — успешный для России, активная помощь союзников. Несмотря на колоссальные усилия российской администрации, потенциал русской промышленности был слишком мал. \\
    Брусиловский прорыв — май–июль 1916 год — произвело впечатление на современников \\
    С \textbf{апреля 1917 года} в первую мировую войны открыто вступают американцы и канадцы — страны тройственного союза обречены на поражение.
\end{enumerate}

\textbf{Февральская революция} начинается в феврале 1917 года — царь уехал в Могилёв, протесты против массовых увольнений на \textbf{Путиловском заводе} — поддерживают работники других предприятий. Массовая забастовка и демонстрации в Петрограде.

\textbf{27 числа} отправленные на разгон демонстрации военные переходят на сторону восставших, отказавшись стрелять в них.

\textbf{1 марта} город был захвачен толпой. Разгром полицейских участков, убийство полицейских.

Временный комитет государственной думы — Временное правительство. Вместе с тем возникает \textbf{Петроградский совет рабочих и солдатских депутатов}. Первый председатель — меньшевик Чхеидзе.

Царь под давлением генералов и своего окружения понимает, что никто не хочет ему подчиняться. Генералы и высшие чиновники делали ставку на \textbf{сохранение монархии с другим монархом}.

\textbf{2-го марта} — отказывается от власти в пользу своего больного сына Алексея — меняет решение и отказывается от власти за себя и за Алексея в пользу своего брата Михаила Александровича, который отказывается принять власть до собрания Учредительного собрания. \textbf{Монархия пала}.

\hfill

В стране возник режим \textbf{Двоевластия}. Правление \textbf{Временного правительства} (кадеты и умеренные эсеры) и \textbf{Петроградского совета рабочих и солдатских депутатов} (меньшевики и эсеры).

Эти органы плохо ладили между собой. Порой издавали противоположные законы, не координировали свои действия. Приказ №1 Петросовета — моральные настроения в армии упали, возрастание числа дезертиров. 

\textbf{Александр Федорович Керенский}, премьер-министр Временного правительства, пытался стабилизировать положение. Вдохновлял солдат. Стал родоначальником технологии \textbf{политического культа}.

\textbf{18 июня} — наступление русской армии под красными революционными знаменами — копировали ритуалы и символы французской революции. Вскоре контрнаступление захлебнулось. Немцы были лучше подготовлены, армия вновь отступила.

\textbf{Июльская демонстрация 3-4 июля} большевиков и анархистов — вся власть советам, долой войну. Демонстрация было жестко подавлена полицией. Страна, несмотря на усилия Керенского, двигалась к гражданской войне.

\textbf{Корниловский мятеж} — точка невозврата, конец августа 1917 года. Керенский испугался. В стране люди стали агрессивными, начались конфликты. Усилилось разрушение институтов. Все усилия Керенского были обречены. Страна стала разваливаться на части. Резкая политизация и радикализация населения. Усиления влияния большевиков. \textbf{Ленин и Троцкий}.

Ленин был главным сторонником вооруженного захвата власти — «апрельские тезисы». Каменев и Зиновьев были против вооруженного захвата, против стремления Ленина захватить власть силой только одной партии. \textbf{Ленинский период} — внутри ВКП(б) господствует система коллективного руководства

Большевики целенаправленно готовятся к захвату власти, готовят отряды красной гвардии.

\hfill

\dots

Керенский уехал в Гатчину искать на фронте верные войска, чтобы пойти с ними на Петроград — конец его политической карьеры.

Кульминационным периодом восстания \textbf{Ленин не руководил}. В Смольном (там располагался Петроградский совет, там было объявлено о совершении революции) после того, как Троцкий объявил, что Временное правительство было арестовано, там появляется Ленин и объявляет, что революция была совершена, Временное правительство было арестовано.

Второй съезд советов рабочих и крестьянских депутатов — большевики боролись за свое влияние. Услышав об аресте Временного правительства, эсеры и кадеты в знак протеста покидают второй съезд, оставляя его большевикам. \textbf{Декрет о мире}, \textbf{декрет о земле} — легализовал все захваты крестьянами помещичьих земель, списали с политической программы партии эсеров, запустили процесс социализации земли — тактическая уступка, оправдали по факту преступные действия крестьян. \textbf{Декрет о власти} передавал всю власть советам. Большевики вскоре заметили, что советы плохо справлялись с своими задачами, например, не могли наладить работу заводов, предприятий — тогда Ленин выступает за привлечение на предприятия старых специалистов: инженеров, технологов. На втором же съезде был выбран состав первого правительства — \textbf{Совет народных комиссаров СССР}.

\textbf{Альянс большевиков с левыми эсерами}.

В \textbf{конце 1917 года — январе 1918 года} — ряд декретов об отмене дореволюционных чинов и званий; законов, дискриминировавших права нацменьшинств; об уравнении прав мужчин и женщин (лишь продолжили тенденцию, заданную Временным правительством). Декрет о восьмичасовом рабочем дне, о всеобщем начальном образовании. Декрет об организации Всероссийской чрезвычайной комиссии во главе с Дзержинским. Январь — два указа, один — о создании Рабоче-крестьянской красной армии, второй — о создании Рабочье-крестьянского военно-морского флота.

Поддерживали организацию выборов \textbf{Учредительного собрания}, надеясь в них победить — большевики на самом деле оказались в меньшинстве.

На следующий день большевики \textbf{силой разогнали} Учредительное собрание. В Петрограде большевики расстреляли демонстрацию рабочих в поддержку Учредительного собрания.

Ленин лукаво написал, что Учредительное собрание отстало от жизни, и что в настоящий момент нужны новые подходы, новые решения, которые могут принять только большевики.

\hfill

\textbf{Брестский мир} — большевики подписали в начале марта 1918 года на антипатриотических условиях — огромные потери земли, флота, выплачивали огромную контрибуцию из всего национализированного имущества (государственного и частного).

\hfill

\textbf{Гражданская война} (1917–1921), активный этап — с 1918 года, 1919 год — для большевиков все висело на волоске. Белые мобилизовали 0.5 млн солдат (вместе с союзниками). К концу гражданской войны красной армии было около 5 млн человек. 

Причины, по которым красным удалось победить: \textbf{несогласованность действий противников}, более профессиональных в военном плане, но никудышных политиков (отказались от помощи Финляндии).

Большевики были лучше организованы, были более гибкими и профессиональными политиками, занимали более выгодную позицию военно-промышленного центра России, стояли в обороне.

Идеологическая слабость — порядок, возвращение собственности, но не справились с этими задачами. Занимались погромами. Опустились до уровня бандитов, действовали столь же авторитарно, как красные.

В Красной армии было много выдвиженцев из низов — импонировало простым людям. Но использовали авторитарные меры — продразверстки, продовольственная диктатура.

Большевики — \textbf{военный коммунизм}. Очень рискованный, практически провальный шаг — коммунизм в военное время. Основные признаки:

\begin{enumerate}
    \item Национализация государством всего имущества
    \item Отмена товарно-денежных отношений
    \item Введение обязательной трудовой повинности
\end{enumerate}

Эта система не справлялась с своими задачами, несмотря на ставку на террор, большевики собирали гораздо меньше продовольствия, чем собирало Временное правительство.

Восстание Кронштадского гарнизона — за советы без коммунистов и за советы без коммунизма. Восстания в Тамбове. Использовали авиацию и химическое оружие.

Замена продразверстки продналогом — более эффективно.

\hfill

\textbf{Новая экономическая политика} — 1921 год. В культурной сфере стало довольно прогрессивным явлением.

1925 год — большевики принимают решение о начале индустриализации. Возникает вопрос о средствах на индустриализацию. НЭП не справляется. Возникает товарный дефицит.

1927 год — Советский союз достиг уровня 1914 года — слишком мало для того времени.

1928 год — коллективизация сельского хозяйства.

\end{document}