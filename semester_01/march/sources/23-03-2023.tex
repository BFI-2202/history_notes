\documentclass{article}
\usepackage[utf8]{inputenc}

\usepackage[T2A]{fontenc}
\usepackage[utf8]{inputenc}
\usepackage[russian]{babel}

\usepackage{multienum}
\usepackage{geometry}

\geometry{
    left=1cm,right=1cm,
    top=2cm,bottom=2cm
}


\title{История}
\author{Лисид Лаконский}
\date{March 2023}

\newtheorem{definition}{Определение}

\begin{document}
\raggedright

\maketitle
\tableofcontents
\pagebreak

\section{История — 23.03.2023}

\subsection{Последствия смутного времени}

\begin{multienumerate}
    \mitemxx{Серьезный экономический упадок}{Потеря влияния и авторитета на международной арене}
    \mitemx{И другое...}
\end{multienumerate}

Рюриковичи дискредитировали себя — \textbf{Михаил Фёдорович Романов} был избран на Земском соборе в 1613-ом году — конец смутного времени.

Уровень жизни населения рос очень медленно.

Множество бунтов: соляной, медный, восстание Степана Разина.

Основные причины: внутренняя политика, юридическое закрепощение крестьян и горожан.

\hfill

Важное событие, усилившее фактор нестабильности — \textbf{церковная реформа патриарха Никона}, целью которой была модификация церковных обрядов и превращение, в перспективе, Москвы в центр православия, подобный Ватикану.

Как считают исследователи, раскольники были наиболее благочестивыми прихожанами своего времени. Негативно восприняли отход от византийских канонов.

\textbf{Колоссальный раскол общества}. Раскольники посчитали, что реформы — признак того, что власть над Россией захватил дьявол. Стремились уйти в полную изоляцию от государства и общества. Собственные лидеры — напр. \textbf{протопоп Авакуум}.

Глубокий конфликт. Земской собор наложил на раскольников анафему.

\hfill

\textbf{Основные причины мятежей}:

\begin{multienumerate}
    \mitemxxx{последствия смуты}{усиление гнета государств}{закрепощение большинства меньшинством}
    \mitemx{частые войны, которое вело государство}
\end{multienumerate}

\hfill

Алексей Михайлович не обладал страстью к проведению масштабных государственных реформ — однако, уже он разрешил одеваться по европейской моде. Интересовался новейшими достижениями европейской науки.

Получил традиционное древнерусское образование. Большая семья, был дважды женат.

\hfill

После смерти Алексея Михайловича на престол вступил его сын \textbf{Федор Алексеевич} (1676–1682), получивший уже европейское образование.

После смерти — \textbf{восстание стрельцов}. Поддерживали \textbf{Софью Алексеевну} в ее притязаниях на престол.

\hfill

\textbf{Петр I} — один из сыновей Алексея Михайловича. С детства интересовался всем военно-морским.

Намного более выгодно смотрелся на фоне своего умственно-отсталого брата Ивана V. Софья также многих не устраивало: русская знать не желала, чтобы женщина возглавляла государство.

Постепенно московские элиты консолидируются вокруг Петра I

\hfill

\textbf{1689 год} — переворот, свержение царевны Софьи.

Однако, пока не управляет страной. Проникается глубокой неприязнью ко всему старорусскому, в принципе не любит Москву.

Рано формируется как последовательный западник.

Приступает к управлению страной. Царь-технократ. Большой интерес к войне, флоту. Тратит большое количество денег на армию.

Политика, направленная на активизацию внешнеполитического статуса России.

\hfill

\textbf{Великое посольство}, одна из целей — заручиться поддержкой европейских государств в борьбе против Османской империи.

Побывал на верфях в Голландии, в Оксфорде и Кембридже в Лондоне. Интересуют знания. В Австрии интересовался краеведческим музеем. Всем технологическим: мануфактуры, обсерватории, больницы, приюты. Абсолютное безразличие к парламентаризму.

\textbf{В политическом отношении был скорее антизападником}. Был сторонником очень консервативных идей — развитие самодержавия, не принимал идеи гражданского общества, парламентаризма.

1770 год — переход на европейское летоисчисление. Радикальный переход на европейские моды, изменения в быту. \textbf{Превращение русских людей в европейцев}.

\hfill

Во время великого посольства меняет планы и решает воевать \textbf{не против Турции, а против Швеции}. Новые союзники — Дания, Саксония, доброжелательный нейтралитет был у Пруссии.

Прерывает великое посольство из-за бунта стрельцов в Москве. Восстание подавили. Жестокие казни. В том числе лично отрубал головы людям.

Был жестоким и злым человеком. Плохие отношения с собственным сыном Алексеем — тот бежал от него в Италию. Пообещал, что сохранит ему жизнь, если он сам вернется в Россию — нарушил слово. Смерть сына не вызвала ни малейшего сожаления.

Прекрасно разбирался в людях в плане поиска талантов. Плохо уживался с родственниками. Первую жену отправил в монастырь, сына казнил. Со второй женой тоже со временем ухудшились отношения.


\hfill

\textbf{Всешутейший, всепьянейший и сумасброднейший собор} — жестоко высмеивал церковь.

Особенность петровской модернизации — проводилась на фоне \textbf{Северной войны}. \textbf{1705 год} — введение рекрутской повинности. Накладывалась на податные сословия. Крайне непопулярная повинность.

Возникновение учебных заведений для офицеров. Постепенно строятся новые предприятия, новые отрасли мануфактурной промышленности. Петр I ищет талантливых людей и ставит их во главе коллегий, дивизий, учебных учреждений.

Основатель наследии Демидовых — кузнец из Тулы, которого Петр I наделил привилегиями.

\hfill

\textbf{1708–1710 год — введение губернской реформы}. Введение элементов рационального государственного управления по европейскому образцу.

\textbf{1711 год} — создание Сената, \textbf{1714 год} — указ о единонаследии, попытка ввести майорат в отношении дворянской недвижимости — замедлить процесс обеднения дворян, создать для дворян стимул учиться и служить.

\textbf{1718–1721} — реформа государственного управления, введение коллегий.

\textbf{1721 год} — объединение церкви с государством. Священный синод. Глава синода — государственный чиновник. В том же году — \textbf{указ, разрешающий предпринимателям и купцам покупать крестьян к фабрикам и заводам}.

Русские предприниматели были связаны с государством. Были обязаны беспрекословно выполнять его заказы. Отличались так же, как и государство, крайне жестоким обращением с людьми.

Отделяет идеи сакральной монархической власти от идеи служения монарха государству.

\textbf{1722 год — введение табели о рангах}. Все дворяне обязаны служить. Наступление светского государства.

\hfill

Роль религии сильно снижается. Реформы снизили авторитет церкви в обществе, особенно среди крестьян — большинство приходских священников жили бедно.

\textbf{1722 год — указ о престолонаследии}, в соответствии с которым Петр I в праве кому угодно завещать престол без необходимости кому-либо об этом отчитываться.

\textbf{1724 год — регламент об учреждении петербургской академии наук}. Последователи Петра I смогли завлечь в Петербург множество видных ученых того времени.

\hfill

\textbf{1721 год — подписание Ништадского мирного договора}, празднование, получение Россией многих европейских территорий. Сенат предлагает Петру I титул императора.

Страна очень громко заявила о себе на международной арене — никто не думал, что Швеция будет побеждена.

Петр I начинает больше заниматься внутренними делами. Заманивает людей в университеты. Имплантирует идею науки — никто не желал учиться.

\hfill

Петр I реализует \textbf{достаточно успешную догоняющую модернизацию}. 

\hfill

Неоднозначен. Люди были расходным материалом. Погибло очень много жителей прифронтовых территорий. Погибло множество строителей Санкт-Петербурга.

\end{document}