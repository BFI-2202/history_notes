\documentclass{article}
\usepackage[utf8]{inputenc}

\usepackage[T2A]{fontenc}
\usepackage[utf8]{inputenc}
\usepackage[russian]{babel}

\usepackage{multienum}
\usepackage{geometry}

\geometry{
    left=1cm,right=1cm,
    top=2cm,bottom=2cm
}


\title{История}
\author{Лисид Лаконский}
\date{March 2023}

\newtheorem{definition}{Определение}

\begin{document}
\raggedright

\maketitle
\tableofcontents
\pagebreak

\section{История — 09.03.2023}

\subsection{Образование единого российского государства в конце 15-го — начале 16-го века}

В \textbf{конце 15-го века} формируется единое российское государство. \textbf{Россия} приходит на смену \textbf{Руси}. Это обеспечило предпосылки для дальнейшего развития русского государства, русской культуры — защищенное государство, обладающее суверенитетом.

\textbf{Исторически прогрессивное явление} — обеспечило сохранение русского народа, русской нации в исторической перспективе.

\hfill

Предпосылками служит длительный период накопления необходимого потенциала, необходимых возможностей.

\hfill

Борьба за средства и деньги между московским и тверским княжествами при \textbf{Даниле Александровиче}, при \textbf{Юрие Даниловиче} (Москва).

Тверское княжество \textbf{доминировало} на Руси — обладало ярлыком Золотой Орды на правление в Владимирском княжестве.

Москва выиграла. Тверской князь убил московского князя без разрешения хана Золотой Орды — тверской князь казнен.

\hfill

В \textbf{1327 году} народ стихийно поднимает восстание в ответ на бесчинство отряда ордынского посланника — мгновенно убивает весь отряд.

Монголы отвечают карательным походом — разграбив Тверь, прошли по соседним княжествам, не тронув московскую землю.

Спустя некоторое время Иван Данилович Калита \textbf{расправляется с тверским князем}. Бежит сначала в Новгород, потом в Псков.

\hfill

С помощью интриг Иван Данилович \textbf{обвиняет тверского князя} в утаивании дани, крамоле — в результате его обвиняют и казнят в Золотой Орде — московский князь \textbf{получает ярлык на великое княжение Владимирское}.

Поддерживает с Золотой Ордой хорошие отношения — агрессивные монгольские набеги прекращаются; экономический подъем.

\hfill

В это время нет никакого общерусского патриотизма, общерусских интересов в сознании людей — противостояние русских княжеств за территории и сферы влияния.

\hfill

Дмитрий Донской воспринимает Мамая как примерно равного ему по статусу, но узурпировавшего власть — в связи с этим \textbf{выступает против него}.

Два года спустя Дмитрий Донской признает власть хана Тохтамыша — платит ему недоимку за два года.

Золотая Орда делает уступку, передает Дмитрию Ивановичу \textbf{ярлык на бессрочное княжение} — Тохтамыш с огромным трудом собрал Золотую Орду, спустя несколько лет терпит поражение от персидского завоевателя Тимура, после которого Орда так и не смогла восстановиться.

\hfill

С ослаблением Золотой Орды \textbf{меняется положение русских земель} — размер дани снижается; она больше не требует участия русских князей в ордынских походах (русские князья относились к этому неоднозначно).

Иногда русские князья в своих интересах использовали поддержку хана. Большинство войн происходило по инициативе самих князей — соревновались в обещаниях большей дани — получали поддержку хана.

\hfill

К середине 15-го века происходят постепенные изменения в менталитете сперва представителей русской элиты, потом русского народа — восприятие Золотой Орды как врагов.

\textbf{Василий II} в итоге побеждает в войне.

\textbf{Великий князь всея Руси Иван III Васильевич} стал диктатором, которого ожидала страна. Знал ошибки своего отца — лучше контролировал себя, не желая их повторять.

\hfill

\textbf{Иван III} сумел успешно решить свои основные задачи. Крайне жестко выколачивает из населения налоги, организовывает самое сильное среди русских княжеств войско с отличной экипировкой и вооружением. Порох, железо, многое другое — экспортируется из-за границы.

Использует традиционный способ расширения своей территории — путем \textbf{завоевания}. Окончательно \textbf{присоединяет новгородское княжество}. Подавляет культуру и традиции Великого Новгорода.

В \textbf{1485-ом году} — \textbf{присоединение тверского княжества} к Москве.

\textbf{1489-ый год} — присоединение огромной вятской боярской республики, богатой ресурсами.

Иван III формирует \textbf{единое государство и новый государственный аппарат} — вводит \textbf{«судебник»}. \textbf{Уничтожает удельную систему} — единственный собственник государства.

Введение системы \textbf{местничества} — системы распределения должностей в зависимости от знатности рода. Имела свои издержки, ограничивала возможности талантливых людей — Иван IV начинает его нарушать при назначении на военные должности, но тем не менее обеспечивало более-менее стабильное функционирование государства.

Иван III придумал \textbf{поместное дворянство, дворянскую конницу — дворянское войско}

\hfill

\textbf{Василий III} продолжает линию, намеченную ее отцом — \textbf{захватывает Псков, Смоленск, Рязанское княжество}. Но скоропостижно умирает в молодом возрасте — \textbf{1533 год}.

\hfill

\textbf{Иван IV} растет испуганным; имел таланты, но не был государственным дельцом.

Оставил страну разоренной.

\subsection{Россия в 16-ом — начале 17-го столетий}

\end{document}