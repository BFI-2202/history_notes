\documentclass{article}
\usepackage[utf8]{inputenc}

\usepackage[T2A]{fontenc}
\usepackage[utf8]{inputenc}
\usepackage[russian]{babel}

\usepackage{multienum}
\usepackage{geometry}

\geometry{
    left=1cm,right=1cm,
    top=2cm,bottom=2cm
}

\title{История}
\author{Лисид Лаконский}
\date{February 2023}

\newtheorem{definition}{Определение}

\begin{document}
\raggedright

\maketitle
\tableofcontents
\pagebreak

\section{История — 09.02.2023}

\subsection{Предмет и методология исторической науки}

Для хронистов и писателей \textbf{средних веков} характерен плюрализм точек зрения о прошлом: с одной стороны, мало интересовались тем, что писалось по истории до них и что писали их современники; с другой стороны, в средние века \textbf{много было сделано для разработки источниковедения}

\hfill

В \textbf{эпоху возрождения} происходит реабилитация античного наследия

\hfill

\textbf{Эпоха просвещения} — представления о том, что человек есть tabula rasa, колоссальная вера в образование, педагогику, способности быстро изменить человечество к лучшему

События французской революции показали, что не все так хорошо — раскол Европы

\hfill

Осмыслением опыта революции стала \textbf{культура романтизма} — зарождение в конце восемнадцатого века в Германии — по первую треть следующего столетия

\textbf{Романтики} были разочарованы в идеях просветителей, считали, что они нуждаются в переосмыслении — гораздо больший интерес историей, возникновение нарративов национальных историй, \textbf{метод историзма}

\begin{definition}
Историзм — научный метод, принцип рассмотрения мира, природных и социально-культурных явлений в \textbf{динамике их изменения, становления во времени, в закономерном историческом развитии}, предполагающий анализ объектов исследования в связи с конкретно-историческими условиями их существования
\end{definition}

Популярность идеологии \textbf{национализма} — девятнадцатый век, особенно вторая половина — расцвет национальных государств

Стимулировал изучение национальной истории, появление национального искусства. Касательно России: \textbf{спор западников и славянофилов}, \textbf{золотой век русской культуры}, \textbf{серебряный век русской культуры}

\hfill

\textbf{История как наука возникает во второй половине девятнадцатого века} — появление специализированных кафедр, ученых специальностей, увеличение количества диссертаций по истории, резкое расширение ученого сообщества, его институционализация

Было более развито рецензирование, чем сейчас. В книжных магазинах могли продаваться даже научные работы студентов

Большим интересом пользовались труды и лекции ученых-историков

До революции в России была сильная школа историков, за рубежом известная как «\textbf{русская школа историков}»

\hfill

Ключевыми фигурами российской исторической науки являются \textbf{Карамзин}, \textbf{Соловьев} и \textbf{Ключевский}.

Пушкин сравнивал Карамзина с Колумбом — написал «\textbf{исторический бестселлер}» в девятнадцатом веке, выдержавший множество изданий

Автор самого объемного в фактологическом отношении труда по российской истории был \textbf{Сергей Михайлович Соловьев}. Кроме того, писал тематические монографии

\hfill

\textbf{Василий Осипович Ключевский} — ученик \textbf{Соловьева}, один из известнейшних русских историков. Совершил переворот в изучении истории России. Первостепенное значение внешним событиям политической истории. \textbf{Курс русской истории} в пяти томах — изучал исторические, социальные и экономические процессы в истории России

Изменение в периодизации истории, новый \textbf{императив — изучение не персоналий, а изучение процессов}. Ключевский - \textbf{позитивист}, служение прошлого непосредственным интересам людей

\hfill

Советское время — \textbf{диктат марксизма-ленинизма}. Крупные фигуры: \textbf{Аарон Яковлевич Гуревич} — \textbf{средневековая история Скандинавии}

\hfill

\textbf{Постсоветский период} — уход навязанной идеологии, перевод иностранных работ на русский язык. Более крупное участие в научном сообществе, но аудитория значительно меньше

\begin{definition}
    \textbf{Объект} изучения истории — общество и человек в истории, а \textbf{сверхзадачей} — поиск модели исторического развития
\end{definition}

История опирается на \textbf{изучение исторических источников} (следов прошлого в настоящем), \textbf{анализ} и \textbf{реконструкцию} на их основе

Другое отличие исторической науке — в ней \textbf{невозможно или практически невозможно поставить эксперимент}, используемые методы: \textbf{историческое наблюдение}, \textbf{исторический анализ}, \textbf{исторический синтез}

Как и в любых науках, \textbf{верификация} своих выводов

\hfill

Процедура изучения любого исторического источника начинается с \textbf{определения подлинности этого источниика} (внешней и внутренней критики), \textbf{внутренняя критика} — попытка понять содержание документа. Чтобы правильно понять, документ всегда ставится в исторический контекст, буквального пониимания слов и предложений недостаточно

Результат — \textbf{научные статьи}, \textbf{монографии}, \textbf{диссертационные исследования}

\subsection{Образование древнерусского государства}

Процесс \textbf{формирования} древнерусского государства \textbf{во второй половине девятого века} — к этому времени существовало не менее пятнадцати \textbf{славянских общностей} — древлянах, кривичей, угличей и так далее — все имели шанс стать отдельными государствами

\textbf{862 год} — приглашение Рюрика, Синеуса и Трувора — мифологическое изложение действительных исторических событий. \textbf{Рюриковичи} — приглашенные правители, но и \textbf{восточные славяне} - автохтонный народ, колонизировавший изначально не свою землю

Сомнение вызывают даты, изложенные в «\textbf{Повести временных лет}». Тем не менее, если обобщить мнения исследователей, процесс формирования государства начинается во второй половине девятого века

\hfill

Феномен исторического сознания восемнадцатого века — \textbf{спор норманистов и антинорманистов} - сегодня ученые исходят из того, что вовсе не признание варягов является признаком образования государства; \textbf{образование государства — сложный и длительный процесс}, в котором призвание варягов является хоть и важным, но отдельным, не ключевым эпизодом

Современная наука \textbf{не интересуется норманской проблемой}. Но, по всей видимости, варяги были призваны с целью экономии ресурсов — славяне считали, что они смогут эффективней выполнять управленческие функции, быть более независимыми

\hfill

\textbf{Ключевое значение в успехе организации единого государства} имело подчинение Киева — \textbf{центра славянского мира} - варяжские князья укрепили свой политический и военный авторитет, что помогло продолжить подчинение и завоевание других славянских объединений

\textbf{Русь} (название государства) — \textbf{русин} (отдельный человек) — \textbf{русь} (группа людей)

Ольга вводит \textbf{областное деление}, вводит \textbf{наместников}, сбор дани в фиксированном размере в специальных местах — для того, чтобы не провоцировать новые конфликты

Сын Ольги, \textbf{Святослав}, покоряет хазар, волжских булгар — с целью простого грабежа, для того, чтобы добиться более удобных условий торговли в отношении русских купцов

\hfill

Походы на \textbf{Предунайскую Болгарию} — \textbf{нарушение} Святославом договоренности с \textbf{Византийским императором}, кроме того вступает в конфликт с греками. \textbf{Князь-воин}, стремящийся к активному территориальному расширению

Становится жертвой печенегов — совершил ошибку, разделив свое войско

\hfill

\textbf{После смерти} Святослава начинается \textbf{борьба его сыновей за великокняжеский престол} — побеждает \textbf{Владимир}

В сложившейся системе статус человека имел зависимость от возраста — \textbf{порождение сильных конфликтов между родственниками после смерти князя}

\end{document}
