\documentclass{article}
\usepackage[utf8]{inputenc}

\usepackage[T2A]{fontenc}
\usepackage[utf8]{inputenc}
\usepackage[russian]{babel}

\usepackage{multienum}
\usepackage{geometry}

\geometry{
    left=1cm,right=1cm,
    top=2cm,bottom=2cm
}

\title{История}
\author{Лисид Лаконский}
\date{February 2023}

\newtheorem{definition}{Определение}

\begin{document}
\raggedright

\maketitle
\tableofcontents
\pagebreak

\section{Русские земли в 13–15 веках}

\subsection{Феодальная раздробленность в Европе и России: общее и особенное}

\subsubsection{История феодальной раздробленности в России}

Феодальная раздробленность на Руси начинается после смерти Ярослава Мудрого в \textbf{1054 году}. Согласно его завещанию, Киевская Русь была разделена во владение между 5 сыновьями:

\begin{multienumerate}
    \mitemxx{\textbf{Изяслав} — Киев и Новгород}{\textbf{Святослав} — Чернигов, Рязань, Муром и Тмутаракань}
    \mitemxxx{\textbf{Всеволод} — Переславль, Ростов}{\textbf{Вячеслав} — Смоленск}{\textbf{Игорь} — Волынь}
\end{multienumerate}

В \textbf{1068 году} общекняжеское войско Ярославичей было \textbf{разбито половцами на реке Альте}. Произошло \textbf{Киевское восстание}, начало конфликтов между братьями и борьбы за Киевский престол. \\[1mm]

В \textbf{1094-1097 годах} — крупная междоусобная война, закончившаяся \textbf{Любечским съездом}, провозгласивший принцип \textbf{«каждый держит вотчину свою»}, фактически означавший, что наследование княжеств происходит не в рамках всей династии Рюриковичей, но в рамках ветви Рюриковичей, за которой было закреплено данное княжество. \\[1mm]

\textbf{Половецкая угроза} сохранялась, удерживала на Руси хоть какое-то единство. После смерти Мстислава в \textbf{1132 году} началась масштабная междоусобица из-за попыток Ярополка нарушить порядок наследования. Изначально Русь \textbf{распалась на 12 крупных княжеств, которые также дробились}. Так, к середине 12-ого века количество таких территориальных образований достигло 50, в 14 веке — 250. \\[1mm]

\subsubsection{Анализ}

\subsection{Объективные и субъективные предпосылки феодальной раздробленности на Руси}

\subsection{Государственность на Руси в период феодальной раздробленности: общее и особенное в развитии русских земель в 12 – начале 13 века}

\subsection{Изменение в политическом строе русских княжеств во второй половине 13 века: политические и социокультурные различия Новгорода, Владимира и западнорусских княжеств}

\subsection{Геополитическая ситуация в мире в начале 13 века: Русь между кочевниками и латинянами}

\subsection{Экономические и политические предпосылки экспансии на Русь с Востока и Запада в начале 13 века}

\subsection{Международное положение русских земель в 13 — середине 14 века}

\end{document}