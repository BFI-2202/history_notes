\documentclass{article}
\usepackage[utf8]{inputenc}

\usepackage[T2A]{fontenc}
\usepackage[utf8]{inputenc}
\usepackage[russian]{babel}

\usepackage{multienum}

\title{История}
\author{Лисид Лаконский}
\date{February 2023}

\newtheorem{definition}{Определение}

\begin{document}
\raggedright

\maketitle
\tableofcontents
\pagebreak

\section{История — 03.02.2023}

Кунц Евгений Владимирович — \textbf{преподаватель, доцент}

\textbf{Электронная почта}: e.v.kunts@mtuci.ru

К следующему семинару следует подготовить первую тему

\hfill

Доклады должны быть продолжительностью не меньше 15 минут, читаться наизусть. \textbf{Структура доклада}:
\begin{enumerate}
    \item \textbf{Введение} — постановка проблемы
    \item \textbf{Основная часть} — раскрытие проблемы
    \item \textbf{Заключение}
\end{enumerate}

\hfill

За \textbf{ответ на вопрос} можно получить от 0 до 7 баллов, за \textbf{уточняющий вопрос} от 0 до 5 баллов, за \textbf{доклад} от 0 до 12 баллов

Опционально: за \textbf{проверку конспекта} (состоящего только из конспекта лекций; один раз, на последнем семинаре) можно получить от 0 до 10 баллов, за \textbf{ответы на вопросы в конце лекции} можно получить от 3 до 4 баллов

\hfill

Подготовка к докладу \textbf{не освобождает} от подготовки к общему заданию. Если набрать 60 и больше баллов, на экзамене дается \textbf{льгота}: сдается один вопрос вместо двух

\end{document}
