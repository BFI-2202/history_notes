\documentclass{article}
\usepackage[utf8]{inputenc}

\usepackage[T2A]{fontenc}
\usepackage[utf8]{inputenc}
\usepackage[russian]{babel}

\usepackage{multienum}
\usepackage{geometry}

\geometry{
    left=1cm,right=1cm,
    top=2cm,bottom=2cm
}

\title{История}
\author{Лисид Лаконский}
\date{February 2023}

\newtheorem{definition}{Определение}

\begin{document}
\raggedright

\maketitle
\tableofcontents
\pagebreak

\section{История — неизвестно}

\subsection{Киевская Русь 9–12 веков}

\subsubsection{Общее и особенное в образовании древнерусского и европейских государств}

Западная цивилизация — \textbf{городская цивилизация}, Древняя Русь развивалась \textbf{в том же русле}.

\textbf{Уровень духовного развития Древней Руси для своего времени был высоким}. Известный английский философ \textbf{Ф. Коплстон} относит зарождение здесь философской мысли к \textbf{первой половине XI века}. Причем справедливо утверждается, что \textbf{Киевскую Русь нельзя отделить от Западной Европы}, истоки философской культуры связывают с выдающимся религиозным философским произведением \textbf{«Слово о законе и благодати»}, автором которого является киевский митрополит Илларион.

\hfill

По сравнению с Европой в основе социально-экономической жизни древнерусского общества лежала \textbf{не частная земельная собственность, а коллективное землевладение свободных крестьян общинников}.

\textbf{Во главе государства стоял наследственный князь, но он не был подлинным государем}. Так как, приезжая в ту или иную волость, князь должен был заключать «ряд» — договор с народным вечем, сохранявшим значительную роль.

Древнерусская знать не обладала необходимыми средствами для подчинения веча. Саботировать ее решения она тоже не могла. \textbf{С помощью веча народ влиял на ход политической жизни в желательном для себя направлении}.

 \hfill

 На Руси \textbf{были вооружены не только князь и дружина, но и рядовое население}. Дружина представляла собой отборное ядро княжеских воинов-телохранителей, постоянных спутников и советников князя, своеобразный «штаб», который во время войны давал народному ополчению командиров, но \textbf{подчинялось ополчение не князю и его мужам, а вечу}.

 \hfill

 В западном понимании \textbf{частной собственности не существовало}, имелись только \textbf{права владения}. \textbf{Носителем верховной власти являлся не великий князь, а княжеский род}. Каждый князь был лишь временным владельцем власти, которая передавалась по принципу \textbf{«старший в роду»}.

 \hfill

В Киевской Руси несколько иной была и роль городов. Если в Европе города - это \textbf{центры торговли, ремесла и культуры}, то на Руси — \textbf{общинные и волостные центры, обладавшие правительственными функциями}, к которым тяготела сельская округа. Города становились основными, порой единственными центрами феодальной государственности.

На Западе город возникает как \textbf{сообщество свободных людей, как гарант их возможности заниматься свободным трудом, и как инструмент защиты их прав}; на Руси же города, кроме Новгорода и Пскова, рождались \textbf{как центры княжеской власти, как форпосты колониальной экспансии, как базы фискально-полицейской деятельности, литургические центры}.

\hfill

Одной из особенностей Древнерусского государства является \textbf{отсутствие рыцарства}. В Западной Европе в середине XI в. завершился процесс внутренней колонизации. \textbf{Ресурсы свободных земель и свободных общинников в Западной Европе практически иссяк к тому времени}. В результате этого в Европе чрезвычайно быстро образовался слой молодых людей, младших детей землевладельцев, все состояние которых часто состояло из щита, копья и коня.

На Руси \textbf{внутренняя колонизация не была завершена ни в XI, ни в ХIII в}. На востоке лежали колоссальные земли вплоть до неведомого тогда Тихого океана. Рыцарство на Руси так и не возникло: младшие дети русских феодалов постоянно могли отъезжать на Восток и находить себе нетронутые земли и не пуганых общинников.

Те же факторы обусловили \textbf{экстенсивный, грабительский характер экономики Древнерусского государства}. На Западе факт исчерпывания ресурсов для внутренней колонизации означал необходимость перехода к интенсивному ведению хозяйства и значит наделения все большими правами производителей, скачала феодалов, как организаторов производства, предоставляя им все больше юридических прав.

\subsubsection{Два центра образования древнерусского государства. Взаимоотношения Киева и Новгорода}

Новгородская Русь была холодной лесной страной. Ее жителям приходилось много и тяжело работать. Люди здесь были суровые и волевые — слабые здесь просто не выживали. Отсюда и \textbf{вече} — сильные духом новгородцы не могли терпеть тирании сильного князя, предпочитали решать все сами. К тому же, новгородцы издревле \textbf{торговали}. Возможно, именно поэтому в домонгольском Новгороде было гораздо больше грамотных, чем в Киеве.

\hfill

Совсем другой образ жизни вели жители Киевской Руси. Здесь другая природа и ландшафт: вместо суровых лесов и холодного моря — бескрайние поля хорошей земли. Поэтому здесь утвердился размеренно-сельский образ жизни. Пожалуй, для простого человека жизнь в киевской Руси была проще. Люди спокойно \textbf{возделывали землю}, предоставив сильному князю право руководить страной. \textbf{Княжья власть здесь была куда сильнее}, а сами князи были амбициознее.

\hfill

В \textbf{882 году}, по летописной хронологии, князь Олег (Олег Вещий), родственник Рюрика, отправился в поход из Новгорода на юг, по пути захватив Смоленск и Любеч, установив там свою власть и поставив на княжение своих людей. В войске Олега были варяги и воины подвластных ему племён — чуди, словен, мери и кривиче. Далее \textbf{Олег с новгородским войском и наёмной варяжской дружиной захватил Киев, убил правивших там Аскольда и Дира и объявил Киев столицей своего государства}.

\subsubsection{Политический и сословный строй Киевской Руси}

Историки по-разному оценивают характер государства данного периода: \textbf{«варварское государство»}, \textbf{«военная демократия»}, \textbf{«дружинный период»}, \textbf{«норманнский период»}, \textbf{«военно-торговое государство»}, \textbf{«складывание раннефеодальной монархии»}.

\hfill

Глава государства носил \textbf{титул князя с почётным определением великий}. Княжеская власть была \textbf{наследственной}. Помимо князей в управлении территориями участвовали великокняжеские \textbf{бояре} и «мужи» — \textbf{дружинники}, нанимавшиеся князем. Бояре также имели свои наёмные дружины, которые в случае необходимости сводились в единое войско.

На местном уровне княжеская власть имела дело с племенным самоуправлением в виде \textbf{веча} и \textbf{«градских старцев»}.

\hfill

В Киевском княжестве бояре для ослабления накала борьбы между княжескими династиями \textbf{поддерживали в ряде случаев соправление князей} и \textbf{даже прибегали к физическому устранению пришлых князей}.

Единственным общерусским политическим органом оставался \textbf{съезд князей}, который решал в основном вопросы борьбы с половцами

\hfill

На основании Русской Правды историки выделяют несколько групп населения Древней Руси. \textbf{Знать} в Древней Руси состояла из выдающихся людей славянских племён и родов, а затем её основную часть составили представители династии Рюриковичей. Их сопровождали \textbf{дружины, из которых позднее сформировалось боярство}. \textbf{Дружина делилась на старшую и младшую}. К числу зажиточных людей относились \textbf{купцы, некоторые ремесленники, а также владельцы крупных земельных участков}.

\hfill

Основное население Руси составляли \textbf{свободные крестьяне, называемые «людинами»}. С течением времени всё больше людей становилось \textbf{смердами — другой группой населения Руси, в которую входили зависимые от князя крестьяне}. Смерды были лично свободны, но подчинялись специальной юрисдикции князя. Смерд, как и обычный человек, в результате пленения, долгов и т. д. мог стать \textbf{челядином (более позднее название — холоп)}. Холопы по сути своей \textbf{являлись рабами и были полностью бесправными}.

В XII веке появились \textbf{закупы — неполные рабы, которые могли выкупить себя из рабства}. Считается, что рабов-холопов на Руси было всё же не так много, однако вполне вероятно, что работорговля процветала в отношениях с Византией.

Русская Правда выделяет также \textbf{рядовичей и изгоев}. \textbf{Первые находились на уровне холопа, а вторые — в состоянии неопределённости} (получившие свободу холопы, изгнанные из общины людины и т. д.), однако находились под защитой церкви. 

\subsubsection{Проблемы социально-экономического развития русских княжеств в 9–12 веках}

О социальной структуре Киевской Руси написано в предыдущем разделе. В данном разделе будет написано об экономике Киевской Руси.

\hfill

Для экономики Древнерусского государства периода 9-12 веков характерен \textbf{ранний феодализм}.
Зарождалась сама основа взаимоотношений между государством, сельским хозяйством и
феодалами. \textbf{Ядром русской экономики в то время считалось именно сельское хозяйство,
которое занимало главенствующее положение}.

Данный период характеризуется \textbf{развитостью товарного хозяйства}, поскольку производилось
почти все необходимое. \textbf{Ремесленное дело} развивалось быстрыми темпами, а \textbf{центрами
становились города}, между тем и в селах развивались отдельные отрасли. Важнейшую роль
играла \textbf{черная металлургия}, поскольку в Древней Руси имелись богатые болотные руды.
Всевозможными способами обрабатывалось железо, из него изготавливались различные
изделия для хозяйственных нужд, военного дела и так далее.

\hfill

Торговля в Древней Руси играла огромное значение, особенно внешняя. \textbf{Внешняя торговля была довольно сильно развита, являлась важной составляющей экономики древнерусских княжеств}. Из Руси \textbf{вывозились на продажу пушнина, воск, мёд, смола, лён и льняные ткани, серебряные вещи, пряслица из розового шифера, оружие, замки, резная кость и прочее}. А предметами ввоза были предметы роскоши, фрукты, пряности, краски и прочее.

\hfill

Формой налогов в Древней Руси выступала \textbf{дань}, которую выплачивали подвластные племена. Чаще всего \textbf{единицей налогообложения выступал «дым», то есть дом, или семейный очаг}. Размер налога традиционно был в \textbf{одну шкурку с дыма}. В некоторых случаях — например, с племени вятичей, — бралось по монете от рала (плуга). \textbf{Формой сбора дани было полюдье}, когда князь с дружиной с ноября по апрель объезжал подданных.

В \textbf{946 году} после подавления восстания древлян княгиня Ольга провела налоговую реформу, \textbf{упорядочив сбор дани}. Она \textbf{отменила полюдье и установила «уроки», то есть размеры дани}, и создала «погосты» — крепости на пути полюдья, в которых жили княжеские управляющие и куда свозилась дань. \textbf{Такая форма сбора дани и сама дань назывались «повоз»}. При уплате налога подданные получали глиняные печати с княжеским знаком, что освобождало их от повторного сбора. Реформа \textbf{содействовала централизации великокняжеской власти и ослаблению власти племенных князей}.

\subsubsection{Законодательство Киевской Руси — «Русская Правда»}

Русская Правда — \textbf{сборник правовых норм} Киевской Руси, датированный различными годами, начиная с 1016 года, \textbf{древнейший русский правовой кодекс}. Является одним из основных письменных источников русского права. Происхождение наиболее ранней части Русской Правды связано с деятельностью князя \textbf{Ярослава Мудрого}. Написана на древнерусском языке. Русская Правда стала основой русского законодательства и сохраняла своё значение до XV—XVI веков.

\paragraph{Право по Русской Правде}

\textbf{Уголовное право}. Как и другие ранние правовые памятники, Русская Правда \textbf{отличает убийство неумышленное}, «в сваде», то есть во время ссоры, \textbf{от умышленного} — «в обиду», и от убийства «в разбое». \textbf{Различалось причинение тяжкого или слабого ущерба, а также действия, наиболее оскорбительные для пострадавшего}

\textbf{Уголовные санкции}

\begin{enumerate}
    \item Правда Ярослава санкционировала \textbf{кровную месть}, но ограничивала круг мстителей определёнными ближайшими родственниками убитого
    \item Штрафы в пользу князя:
    \begin{enumerate}
        \item \textbf{Вира} — штраф за убийство свободного человека
        \item \textbf{Полувирье} — штраф за тяжкие увечья свободному человеку
        \item \textbf{Продажа} — штраф за другие уголовные преступления — нанесение менее тяжких телесных повреждений, кражу и другое
    \end{enumerate}
    \item Плата пострадавшим:
    \begin{enumerate}
        \item \textbf{Головничество} — плата в пользу родственников убитого
        \item \textbf{Плата «за обиду»} — как правило, плата потерпевшему
        \item \textbf{Урок} — плата хозяину за украденную или испорченную вещь или за убитого холопа
    \end{enumerate}
    \item Наиболее тяжкими преступлениями считались \textbf{разбой «безъ всякоя свады», поджог гумна или двора и конокрадство}. За них преступник подвергался потоку и разграблению. Первоначально это была высылка преступника и конфискация имущества, позднее — \textbf{преступник и его семья обращались в рабство, а его имущество подвергалось разграблению}. Поток и разграбление инициировала община, а осуществляла княжеская власть, то есть \textbf{эта мера наказания уже была поставлена под контроль государства}
\end{enumerate}

\hfill

\textbf{Частное право.} По Русской Правде купец мог \textbf{отдавать имущество на хранение} (поклажа). Совершались \textbf{ростовщические операции}: в рост давались деньги — отданное (исто) возвращалось с процентами (резы), или продукты с возвратом в пропорционально большем размере. Подробно представлены нормы наследственного права. \textbf{Предусматривалось наследование как по закону, так и по завещанию}

\hfill

\textbf{Процессуальное право.} \textbf{Уголовные правонарушения рассматривал княжий (княжеский) суд} — суд, осуществлявшийся представителем князя. Пойманного на дворе в ночное время вора можно было убить на месте или вести на княжий суд.

По гражданским делам процесс носил состязательный (обвинительный) характер, при котором \textbf{стороны были равноправными и сами осуществляли процессуальные действия}.

\paragraph{Социальные категории по Русской Правде}

\textbf{Знать и привилегированные слуги}

\begin{enumerate}
    \item \textbf{Знать в Русской Правде представлена князем и его старшими дружинниками — боярами}. Князю идут штрафы, имущество которого защищают некоторые статьи, и именем которого вершится суд.
    \item \textbf{Привилегированное положение имели тиуны, огнищане — высокопоставленные княжеские и боярские слуги}, а также княжеский старший конюх
\end{enumerate}

\hfill

\textbf{Рядовые свободные жители}

\begin{enumerate}
    \item Основное действующее лицо Русской Правды — \textbf{муж — свободный мужчина}
    \item Русин — житель Киевской Руси; \textbf{дружинник: гридин — представитель боевой дружины}
    \item \textbf{Купчина} — дружинник, занимавшийся торговлей
    \item \textbf{Ябетник} — дружинник, связанный с судебным процессом
    \item \textbf{Мечник} — сборщик штрафов
    \item \textbf{Изгой} — человек, потерявший связь с общиной
\end{enumerate}

\hfill

\textbf{Зависимое население.} Привилегированное положение среди зависимых людей имели \textbf{княжеские кормильцы, а также княжеские сельские и ратайные старосты}

Низшее положение занимали \textbf{смерды, холопы, рядовичи и закупы}

\begin{enumerate}
    \item \textbf{Смерд} — крестьянин, в этом контексте зависимый крестьянин
    \item \textbf{Холопство} могло быть \textbf{обельным} (полным) или \textbf{закупным}. Обель — пожизненный раб. Женский род — роба.
    \item \textbf{Закуп} — свободный человек, взявший купу — кредит, и попавший в зависимость до тех пор, пока не отдаст или не отработает этот долг
    \item \textbf{Рядович} — лицо, поступившее на службу и ставшее зависимым по «ряду», то есть договору
\end{enumerate}

\end{document}