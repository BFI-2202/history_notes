\documentclass{article}
\usepackage[utf8]{inputenc}

\usepackage[T2A]{fontenc}
\usepackage[utf8]{inputenc}
\usepackage[russian]{babel}

\usepackage{multienum}
\usepackage{geometry}

\geometry{
    left=1cm,right=1cm,
    top=2cm,bottom=2cm
}


\title{История}
\author{Лисид Лаконский}
\date{April 2023}

\newtheorem{definition}{Определение}

\begin{document}
\raggedright

\maketitle
\tableofcontents
\pagebreak

\section{История — 06.04.2023}

\subsection{Россия и мир в первой половине девятнадцатого столетия}

Россия вступает в начало девятнадцатого века с последним в истории страны дворцовым переворотом в ночь 11 на 12 марта 1801 года — убийство российского императора Павла I.

Александр I Павлович был замешан в заговоре — будущий император. Об участии великих князей в заговоре было известно даже за рубежом.

\hfill

Век, в значительной степени сформировавший русское общество и русское мировоззрение. Девятнадцатый век — век формирования общества модерна. Управляется по совершенно иным законам, нежели феодальное общество: предприниматели, владельцы капитала, наемные работники. Большую роль начинает играть наука — одна из опор экономической сферы.

Повышенный спрос на образованных людей. Развитие и диверсификация системы образования.

Рождение модерна сопровождалось целой чередой социальных революций. В Европе их центром продолжает оставаться Франция. После французской революции страна находится в поиске себя — чередуются гибридные режимы: ограниченная монархия, авторитарный режим, и, наконец, — республика.

Страной-авангардом, «мастерской мира» и «владычицей морей» является Великобритания. В ней впервые формируется модерная политическая система (парламент) и модерная судебная система (с институтом присяжных).

\hfill

Рождение Соединенных Штатов Америки, «американские реалии — это наше будущее; американская система демократии со временем завоюет мир». Сформировались в ходе войны за независимость как исходно модернизированное общество.

Продолжение эры мировых стилей в области культуры: барокко (вторая половина 17 века), классицизм, ...

\hfill

Французская революция вызвала серьезнейшую реакцию на раскол в современном мире. Раскол Европы. Наполеоновские войны — на всех территориях Наполеон вводит свой кодекс.

Опосредованное влияние — на русских — 1813–1815-ые годы — заграничный поход русской армии. Поколение декабристов и декабристов без декабря. Влияние Европы на русских интеллектуалов.

В первой половине 19 века, тем не менее, преобладает консервативная реакция на последствия революции — романтизм (возникновение в конце 18 века), реакция против универсалистских идей эпохи европейского просвещения. Национальные истории, конструирование идеи нации. Народ — нация — является источником прогресса.

Представители: Байрон, Жуковский, ранний Пушкин, Лермонтов, Марлинский. В Европе — Фридрих Шиллер, Иоганн Вольфганг Гете. Полное самовыражение. Идеи уникальности, индивидуальности личности и отдельного народа. Путь у каждого народа особенный.

Идеология национализма. Последствия — рост милитаризма в Европе, колоссальный прогресс в сфере вооружений. Зенит идеи национального государства — вторая половина девятнадцатого века.

В настоящее время идет процесс разложения национального государства.

\hfill

Другое влияние девятнадцатого века — зенит научной революции. Открытия Дарвина — колоссальный удар по роли религии в обществе. Открытия Менделеева, Максвелла, Резерфорда, Фарадея, Столетова.

Стимулирование научно-технического прогресса. Трансформация жизни людей и внешнего облика городов.

Первая великая информационная революция — становятся более дешевыми газеты, журналы, книги. Услуги почты становятся дешевыми и эффективными.

\hfill

Девятнадцатый век — распространение образования в обществе — внедрение законов о всеобщем начальном образовании.

Прогрессирующая урбанизация. Рост городов, городского населения. Являются центрами культуры, научных и технологических инноваций.

Маркс, Энгельс — «Капитал», Фридрих Ницше — ницшеанство, позитивизм и сциентизм.

Уровень свободы в Франции, США, Великобритании достигает очень высокой точки.

\hfill

Александр I, 1801 — 1825 год, Россия — неразрывная часть Европы. Создание новой тотальной идеологии изоляционизма страны.

Приходит к власти с сформировавшейся политической программой. Хорошее воспитание. Фредерик Сезар Лагарп привил идеи либерализма будущему императору.

Всю жизнь Александр I оставался убежденным либералом.

Победил наполеоновскую Францию, возвел Россию на беспрецедентную внешнеполитическую высоту.

\hfill

Николай I — апогей самодержавия. Проигрывает Крымскую войну.

\hfill

Первые четыре года Александр I идет навстречу обществу.

Ряд важных законов: разрешение недворянам владеть недвижимостью — поощрение развития предпринимательства. За дворянами оставалась привилегия владения крепостными.

Появление довольно большого количества женщин-предпринимательниц.

1802-ой год — указ о создании министерств, окончательное упразднение коллегий. Принцип единоначалия. Первый прецедент в мире создания министерства народного просвещения.

1803-й год – 1804-й год — модернизация системы высшего и среднего образования в России. Открытие новых университетов в России, повышение их финансирования государством.

Новые университеты: Виленский университет, университет в Дерпте, педагогический институт в Петербурге, Казанский университет, Харьковский императорский университет, преобразования главного педагогического университета в Санкт-Петербургский университет.

В 1804 году Московский императорский университет получает устав.

Университеты управляли гимназиями, городскими училищами, приходскими начальными училищами, всеми частными учебными заведениями.

\hfill

Михаил Михайлович Сперанский — второе лицо государства при Александре I.

Судебная — правительствующий сенат, исполнительная — совет министров, законодательная — государственная дума.

1814-й год — губернатор в Пензе, затем — губернатор в Западной Сибири.

1822-й год — Александр I возвращает Сперанского в Санкт-Петербург.

\end{document}